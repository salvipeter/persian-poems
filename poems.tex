\poem{Shâh Ne`matollâh Vali}{شاه نعمت الله ولی}{
صبح و سحر و بلبل و گلزار یکیست & معشوقه و عشق و عاشق و یار یکیست\\
هر چند درون خانه را می‌نگرم & خود دایره و نقطه و پرگار یکیست
}{
sobh-ō sahar-ō bolbol-o golzâr yek-i-st\\
ma`shuqe-o `eshq-o `âsheq-ō yâr yek-i-st\\
har chand darun-e khâne-râ minegaram\\
khod dâyere-ō noqte-o pargâr yek-i-st
}{
Éj, hajnal, bülbül és virágtő -- mind egy;\\
vágy és szerelem, barát-barátnő -- mind egy.\\
Bárhányszor nézem én a Házat s mit rejt:\\
Én és a Kör és a Pont s a Körző -- mind egy.
}{
A Pont (vagy ,,a Kör pontja'') Mohamedre utal; a Kör a világ, a Körző a Teremtő.
A bülbül reménytelen epekedése a rózsa iránt gyakori motívum.
}
\poem{Abulvafâ Khârazmi}{ابو الوفا خوارزمی}{
بر عقل چو کشف پرده‌ها بود محال & عقل از پس پرده کرد از عشق سؤال\\
تا هست رونده هستی اوست حجاب & ور نیست شود که بهره یابد ز وصال
}{
bar `aql cho kashf-e pardehâ bud mahâl\\
`aql az pas-e parde kard 'az `eshq soâl\\
tâ hast ravande hasti-yē u-st hejâb\\
v-ar nist shavad ke bahre yâbad ze vasâl
}{
Nem képes Ész a leplen átlátni, s ezért\\
függöny mögül ő útmutatást Vágytól kért:\\
,,Falként fedi el léte a vándort, de ha ő\\
nincsen, a haszon kié, ha céljához elért?''
}{
Abulvafâ Khârazmi XV.~századi költő, szúfi filozófus.
}

\poem{Hâfez}{حافظ}{
چون جامه ز تن برکشد آن مشکین خال & ماهی که نظیر خود ندارد به کمال\\
در سینه ز نازکی دلش بتوان دید & مانندهٔ سنگ خاره در آب زلال  
}{
chon jâme ze tan bar-kashad ân moshkin khâl\\
mâh-i ke nazir-e khod nadârad be kamâl\\
dar sine ze nâzoki del-ash betvân did\\
mânande-ye sang-e khâre dar 'âb-e zolâl
}{
Mint páratlan szépségű hold, olyan ő,\\
arcán anyajeggyel, ha levetkőzik e nő.\\
Mellén, bizony oly törékeny, átsejlik a szív,\\
mint kristálytiszta vízen egy gránitkő.
}{
}
