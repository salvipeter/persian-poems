%% -*- bidi-paragraph-direction: left-to-right -*-
\poem{Rudaki}{رودکی}{
جز حادثه هرگز طلبم کس نکند & یک پرسش گرم جز تبم کس نکند\\
ور جان به لب آیدم بجز مردم چشم & یک قطرهٔ آب بر لبم کس نکند
}{
joz hâdese hargez talab-am kas nakonad\\
yek porsesh-e garm joz tab-am kas nakonad\\
v-ar jân be lab âyadam be joz mardom-e cheshm\\
yek qatre-ye âb bar lab-am kas nakonad
}{
Már látogatóba nem jön el más, csak a gond,\\
s nincs más, csak a láz, mely meleg üdvözlést mond.\\
Egy csepp vizet ajkamra nem ad más, csak a szem,\\
lelkem mikor eltávozik, és könnyeket ont.
}{
}
\poem{Mahsati Ganjavi}{مهستی گنجوی}{
چشمم چو به چشم خویش چشم تو بدید & بی چشم تو خواب چشمم از چشم پرید\\
ای چشم همه چشم به چشمت روشن & چون چشم تو چشم من دگر چشم ندید
}{
cheshm-am cho be cheshm-e khish cheshm-ē to bedid\\
bi cheshm-e to khâb-e cheshm-am az cheshm parid\\
ey cheshm hamē cheshm be cheshm-et rowshan\\
chon cheshm-e to cheshm-e man degar cheshm nadid
}{
Láttam szemedet saját szememmel, s emiatt\\
estére szememből álmom messze szaladt.\\
Ó szem, mire minden szemnek fénye vetül,\\
nincs más szempár sehol, szemem min megakad.
}{
}
\poem{Ruzbehân Baqli}{روزبهان بقلی}{
دی آینهٔ خویش به صیقل دادم & روشن کردم به پیش خود بنهادم\\
در آینه عیب خویش چندان دیدم & کز عیب کسان هیچ نیامد یادم
}{
di âyene-yē khish be seyqal dâdam\\
rowshan kardam be pish-e khod benhâdam\\
dar 'âyene `eyb-e khish chandân didam\\
k-az `eyb-e kasân hich nayâmad yâd-am
}{
Tegnap lecsiszoltam ezt a tükröt szépen,\\
míg fényes nem lett, s ahogy állt egy széken,\\
rápillantván annyi hibámat láttam,\\
hogy nem jut eszembe most a másé éppen.
}{
}
\poem{`Eynolqozât Hamadâni}{عین‌القضات همدانی}{
ناگه ز درم درآمد آن دلبر مست & جام می لعل نوش کرد و بنشست\\
از دیدن و از گرفتن زلف چو شست & رویم همه چشم گشت و چشمم همه دست
}{
nâgah ze dar-am dar-âmad ân delbar-e mast\\
jâm-ē mey-e la'l nush kard-ō benshast\\
az didan-o az gereftan-ē zolf cho shast\\
ruy-am hame cheshm gasht-o cheshm-am hame dast
}{
Ő hirtelen ajtóm küszöbén átlépett\\
s bort íva magához húzott egy széket.\\
Gyűrűző fürtjét hogy lássam s fogjam,\\
arcom csupa szem, szemem pedig mind kéz lett.
}{
}
\poem{Rashidoddin Vatvât}{رشید‌الدین وطواط}{
بویت شنوم ز باد بی‌هوش شوم & نامت شنوم ز خلق مدهوش شوم\\
اول سخنم تویی چو در حرف آیم & واندیشهٔ من تویی چو خاموش شوم
}{
buy-et shenavam ze bâd bihush shavam\\
nâm-et shenavam ze khalq madhush shavam\\
avval sokhan-am to-i cho dar harf âyam\\
v-andishe-ye man to-i cho khâmush shavam
}{
Szél illatodat ha hozza, mámorba esem;\\
mástól nevedet ha hallom, elvesztem eszem.\\
Szólok -- s első szavam te vagy; nem szólok --\\
s csak téged idéz némán emlékezetem.
}{
}
\poem{Rumi}{رومی}{
خورشید و ستارگان و بدر ما اوست & بستان و سرای و صحن و صدر ما اوست\\
هم قبله و هم روزه و صبر ما اوست & عید و رمضان و شب قدر ما اوست
}{
khorshid-o setâregân-o badr-ē mâ u-st\\
bostân-o sarây-o sahn-o sadr-ē mâ u-st\\
ham qeble-o ham ruze-o sabr-ē mâ u-st\\
`id-ō ramazân-ō shab-e qadr-ē mâ u-st
}{
Nap, csillagok és Hold, tündöklő -- ez is Ő.\\
Udvar, kert, otthon, s asztalfő -- ez is Ő.\\
Böjt és türelem, kibla, a szent kő -- ez is Ő.\\
Kadr-éj Ramadánkor, ó, ha eljő -- ez is Ő.
}{
}
\poem{Sa`di}{سعدی}{
آن یار که عهد دوستداری بشکست & می‌رفت و منش گرفته دامان در دست\\
می‌گفت دگر باره به خوابم بینی & پنداشت که بعد از آن مرا خوابی هست
}{
ân yâr ke `ahd-e dustdâri beshekast\\
miraft-o man-esh gerefte dâmân dar dast\\
migoft degar bâre be khâb-am bini\\
pendâsht ke ba`d az ân ma-râ khâb-i hast
}{
Egy nap, mikor esküvéseit szegve a párom\\
elment, szólt, hogy nem érdemes rá várnom:\\
,,Álmodban fogsz csak engem újból látni!''\\
Csak nem hiszi, hogy szememre jön még álom?
}{
}
\poemnotoc{Sa`di}{سعدی}{
آنان که پری‌روی و شکرگفتارند & حیفست که روی خوب پنهان دارند\\
فی‌الجمله نقاب نیز بی‌فایده نیست & تا زشت بپوشند و نکو بگزارند
}{
ânân ke pari-ruy-o shekar-goftâr-and\\
heyf-ast ke ruy-e khub penhân dârand\\
feljomle neqâb niz bi-fâyede nist\\
tâ zesht bepushand-o neku bogzârand
}{
Mily kár, hogy a sok tündér utcára ha lép,\\
arcát mind elrejti, s nem látja a nép.\\
Van haszna a fátyolnak azért mégis tán:\\
hogy felvegye az, ki csúnya, s eldobja, ki szép.
}{
}
\poem{Kamâloddin Esmâ`il}{کمال‌الدین اسماعیل}{
هر شب ز غمت تو ای نگارین یارم & وز مهر رخ چو ماهت ای دلدارم\\
تا وقت سحر به ماه در می‌نگرم & وز دیده ستارگان فرو می‌بارم
}{
har shab ze qam-et to ey negârin yâr-am\\
v-az mehr-ē rokh cho mâh-et ey deldâr-am\\
tâ vaqt-e sahar be mâh dar-minegaram\\
v-az dide setâregân foru mibâram
}{
Szép kedvesem, arra vágyom, éljél te velem:\\
hold-arcod az ok, hogy éjjel én rendszeresen\\
hajnalhasadásig csak a Holdat nézem,\\
és csillagokat záporozik tőle szemem.
}{
}
\poem{`Alâ'oddowle Semnâni}{علاءالدوله سمنانی}{
وقت سحری در‌آمد آن مهرویم & در گوش دلم گفت که ای دلجویم\\
تو هیچ مباش تا همه من باشم & تو هیچ مگوی تا همه من گویم
}{
vaqt-ē sahar-i dar-âmad ân mahru-yam\\
dar gush-e del-am goft ke ey delju-yam\\
tō hich mabâsh tâ hamē man bâsham\\
tō hich maguy tâ hamē man guyam
}{
Hold-arcú kedvesem szobámban jár-kel,\\
szívemhez szól, fülembe súg, így ráz fel:\\
,,Légy semmi, hogy én váljak a mindenséggé;\\
s hogy mindent elmondjak, hallgassál el.''
}{
}
\poemnotoc{`Alâ'oddowle Semnâni}{علاءالدوله سمنانی}{
صد خانه اگر به طاعت آباد کنی & به زان نبود که خاطری شاد کنی\\
گر بنده کنی ز لطف آزادی‌را & بهتر که هزار بنده آزاد کنی
}{
sad khâne agar be tâ`at âbâd koni\\
beh z-ân nabovad ke khâter-i shâd koni\\
gar bande koni ze lotf 'âzad-i-râ\\
behtar ke hazâr bande âzâd koni
}{
Építhetsz szentélyt, száz fajtát-formát:\\
egy kedves szócska többet ér majd odaát.\\
Jobb egy szabad embert leigáznod keggyel,\\
mint felszabadítanod ezer rabszolgát.
}{
}
\poem{Hâfez}{حافظ}{
چون جامه ز تن برکشد آن مشکین خال & ماهی که نظیر خود ندارد به کمال\\
در سینه ز نازکی دلش بتوان دید & مانندهٔ سنگ خاره در آب زلال  
}{
chon jâme ze tan bar-kashad ân moshkin khâl\\
mâh-i ke nazir-e khod nadârad be kamâl\\
dar sine ze nâzoki del-esh betvân did\\
mânande-ye sang-e khâre dar 'âb-e zolâl
}{
Mint páratlan szépségű hold, olyan ő,\\
arcán anyajeggyel, ha levetkőzik e nő.\\
Mellén, bizony oly törékeny, átsejlik a szív,\\
mint kristálytiszta vízen egy gránitkő.
}{
}
\poem{Shâh Ne`matollâh Vali}{شاه نعمت‌الله ولی}{
صبح و سحر و بلبل و گلزار یکیست & معشوقه و عشق و عاشق و یار یکیست\\
هر چند درون خانه را می‌نگرم & خود دایره و نقطه و پرگار یکیست
}{
sobh-ō sahar-ō bolbol-o golzâr yek-i-st\\
ma`shuqe-o `eshq-o `âsheq-ō yâr yek-i-st\\
har chand darun-e khâne-râ minegaram\\
khod dâyere-ō noqte-o pargâr yek-i-st
}{
Éj, hajnal, bülbül és virágtő -- mind egy;\\
vágy és szerelem, barát-barátnő -- mind egy.\\
Bárhányszor nézem én a Házat s mit rejt:\\
Én és a Kör és a Pont s a Körző -- mind egy.
}{
A Pont (vagy ,,a Kör pontja'') Mohamedre utal; a Kör a világ, a Körző a Teremtő.
A bülbül reménytelen epekedése a rózsa iránt gyakori motívum.
}
\poem{Abolvafâ Khârazmi}{ابو‌الوفا خوارزمی}{
بر عقل چو کشف پرده‌ها بود محال & عقل از پس پرده کرد از عشق سؤال\\
تا هست رونده هستی اوست حجاب & ور نیست شود که بهره یابد ز وصال
}{
bar `aql cho kashf-e pardehâ bud mahâl\\
`aql az pas-e parde kard 'az `eshq soâl\\
tâ hast ravande hasti-yē u-st hejâb\\
v-ar nist shavad ke bahre yâbad ze vasâl
}{
Nem képes Ész a leplen átlátni, s ezért\\
függöny mögül ő útmutatást Vágytól kért:\\
,,Falként fedi el léte a vándort, de ha ő\\
nincsen, a haszon kié, ha céljához elért?''
}{
Abolvafâ Khârazmi XV.~századi költő, szúfi filozófus.
}
\poemnotoc{Abolvafâ Khârazmi}{ابو‌الوفا خوارزمی}{
آمد بر من خیال او نیم شبان & گفتم که نثار پای تست این دل و جان\\
گفتا چه دل و چه جان ترا ملک کجاست & آسان باشد سخاوت از مال کسان
}{
âmad bar man khayâl-e u nim-e shabân\\
goftam ke nesâr-e pâ-ye to-st in del-o jân\\
goftâ che del-ō che jân to-râ melk kojâ-st\\
âsân bâshad sakhâvat az mâl-e kasân
}{
Álomkép jött el éjjel, ébenfa-hajú,\\
mondtam, hogy ,,E szívet s lelket vidd, te fiú!''\\
Szólt: ,,Mely szív és lélek? Hol van, mi tiéd?\\
Könnyen vagy a másokéval ily nagyvonalú.''
}{
}
\poem{Jâmi}{جامی}{
دیدار تو ای یار پسندیدهٔ من & حیف است بدین دیدهٔ غمدیدهٔ من\\
در دیدهٔ من نشین و بغشای نقاب & خود بین رخ خویش لیکن از دیدهٔ من
}{
didâr-e to ey yâr-e pasandide-ye man\\
heyf-ast bed-in dide-ye qamdide-ye man\\
dar dide-ye man neshin va-bogshây neqâb\\
khod bin rokh-e khish liken az dide-ye man
}{
Szépséged, kedves, kivetett rám sarcot,\\
csüggedt szemeim már nem akarnak harcot.\\
Ülj hát a szemembe, vesd le kendőd, s te magad\\
bámuld a szemem lencséjén át arcod.
}{
Jâmit (1492-1414) a középkori perzsa irodalom utolsó nagy költőjeként tartják számon.
Verses és prózai művei mellett számos filozófai, teológiai és nyelvészeti tárgyú könyvet is írt.
}
