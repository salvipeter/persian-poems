%% -*- bidi-paragraph-direction: left-to-right -*-
\poem{Rudaki}{رودکی}{
جز حادثه هرگز طلبم کس نکند & یک پرسش گرم جز تبم کس نکند\\
ور جان به لب آیدم بجز مردم چشم & یک قطرهٔ آب بر لبم کس نکند
}{
joz hâdese hargez talab-am kas nakonad\\
yek porsesh-e garm joz tab-am kas nakonad\\
v-ar jân be lab âyadam be joz mardom-e cheshm\\
yek qatre-ye âb bar lab-am kas nakonad
}{
Már látogatóba nem jön el más, csak a gond,\\
s nincs más, csak a láz, mely meleg üdvözlést mond.\\
Egy csepp vizet ajkamra nem ad más, csak a szem,\\
lelkem mikor eltávozik, és könnyeket ont.
}{
\emph{Rudaki\/}t (859{--}940) az újperzsa nyelv első
nagy költőjeként tartják számon. Hozzá fűződik
a versek \emph{divân\/}ba rendezése is.
}
\poem{`Onsori}{عنصری}{
گفتم که چرا چو ابر خون بارانم & گفت از پی آن که من گل خندانم\\
گفتم که چرا بی تو چنین پژمانم & گفت از پی آن که تو تنی من جانم
}{
goftam ke cherâ cho abr khun bârânam?\\
goft az pey-e ân ke man gol-ē khandân-am\\
goftam ke cherâ bi to chonin pezhmân-am?\\
goft az pey-e ân ke tō tan-i man jân-am
}{
,,Vért'' -- kérdte -- ,,miért ontok, mint egy felleg?''\\
,,Mert rózsa vagyok'' -- felelt -- ,,virítok s döflek.''\\
,,Mondd akkor: nélküled miért csüggedek én?''\\
,,Mert'' -- mondta -- ,,te test vagy, én azonban lélek.''
}{
\emph{`Onsori} (kb.~961{--}1039) udvari költő,
ódáit az ékesszólás és a formai tökély jellemzi.
Műveinek csak egy töredéke maradt ránk.
}
\poem{Abu Sa'id Abolkheyr}{ابوسعید ابوالخیر}{
جسمم همه اشک گشت و چشمم بگریست & در عشق تو بی‌جسم همی باید زیست\\
از من اثری نماند این عشق ز چیست & چون من همه معشوق شدم عاشق کیست
}{
jesm-am hame ashk gasht o cheshm-am begrist\\
dar `eshq-e to bi-jesm hami bâyad zist\\
az man 'asar-i namânad in `eshq ze chi-st\\
chon man hame ma`shuq shodam `âsheq ki-st
}{
Könnyé vált testem, s elsírtam mindet,\\
test nélkül kell élnie annak, ki szeret.\\
Honnan jön e vágy, mikor nyomom sincs itt már?\\
Eggyé váltunk -- most a Szerelmes ki lehet?
}{
\emph{Abu Sa'id Abolkheyr} (967{--}1049) szúfi gondolkodó.
A mai türkmén határ közelében élt, de híre Spanyolországig eljutott.
Tanítása meghatározó volt a szúfi miszticizmusban,
de ő maga nem írt jelentősebb filozófiai művet.

Jól ismerte Avicennát (\emph{Ebn Sinâ}):
több találkozásukról is írt az önéletrajzában,
és néhány vele váltott levele is fennmaradt.
}
\poem{Qatrân Tabrizi}{قطران تبریزی}{
ای زلف تو از رخان من پرچین‌تر & وز خون دو چشم من رخت رنگین‌تر\\
هر روز تو نیکوتر و من زارترم & هر روز تو دلبرتر و من بی‌دین‌تر
}{
ey zolf-e to az rokhân-e man por-chintar\\
v-az khun-e do cheshm-e man rokh-et rangintar\\
har ruz to nikutar-o man zârtar-am\\
har ruz to delbartar-o man bi-dintar
}{
Arcomnál fodrosabb a dús hajfonatod,\\
vérző szemeimnél pirosabb még arcod.\\
Szebb vagy minden nap, s nyomorultabb vagyok én,\\
és egyre hitetlenebb, ha látom alakod.
}{
\emph{Qatrân Tabrizi} (1009{--}1072) azeri költő. Elsősorban ódáiról és
dicshimnuszairól híres, melyeket kb.~30 különböző mecénáshoz írt.
}
\poem{`Omar Khayyâm}{عمر خیام}{
آن قصر که با چرخ همی زد پهلو & بر درگه آن شهان نهادندی رو\\
دیدیم که بر کنگره‌اش فاخته‌ای & بنشسته همی گفت که کوکوکوکو
}{
Ân qasr ke bâ charkh hami zad pahlu\\
Bar dargah-e ân shahân nehâdandi ru\\
Didim ke bar kongere-ash fâkhte-i\\
Benshaste hami goft ke kukukuku
}{
Ez volt az az Éggel versengő palota,\\
sok sah kapujában hajlongott valaha.\\
Most egy madarat láttam a csipkés bástyán --\\
ott ült s így szólt: ,,Kakukk! Kakukk! Minden oda!''
}{
\emph{`Omar Khayyâm} (1048{--}1131) matematikus, csillagász és költő.
Versei a XIX.~század második felében Európa-szerte ismertek lettek Edward
FitzGerald átköltésein keresztül.
}
\poemnotoc{`Omar Khayyâm}{عمر خیام}{
از آمدن بهار و از رفتن دی & اوراق وجود ما همی گردد طی\\
می خور مخور اندوه که فرمود حکیم & غمهای جهان چو زهر و تریاقش می
}{
az 'âmadan-ē bahâr-o az raftan-e dey\\
owrâq-e vojud-e mâ hami gardad tey\\
mey khor makhor anduh ke farmud hakim\\
qamhâ-ye jahân cho zahr-o teryâq-esh mey
}{
Már elmegy a tél, jő a tavasz, rajta a sor,\\
apránként így lapozza létünket a Kor.\\
Koccints, ne szomorkodj, hisz' az orvos mondta:\\
méreg minden gond, s rá ellenszer a bor.
}{
Tudományos munkáiban többek közt geometriai megoldást adott
kúpszeletek metszésére, és megalkotta a \emph{Jalâli} naptárat,
melynek változatai ma is használatban vannak Iránban és
Afganisztánban.  Foglalkozott a zenei skálák matematikai
rendszerezésével is.
}
\poem{Abolfazl Rashidoddin Meybodi}{ابو‌الفضل رشیدالدین میبدی}{
ز اول که مرا عشق نگـارم نو بود & همسایه به شب ز نالهٔ من نغنود\\
کم گشت کنون ناله که عشقم بفزود & آتش چو همه گرفت کم گردد دود
}{
z-avval ke ma-râ `eshq-e negâr-am now bud\\
  hamsâye be shab ze nâle-yē man naqonud\\
kam gasht konun nâle ke `eshq-am bofzud\\
  âtesh cho hamē gereft kam gardad dud
}{
Nem hagytam a szomszédnak régen pihenést,\\
úgy sóhajtoztam este, vágyván ölelést.\\
Csökkentek a sóhajok, szerelmem hogy nőtt:\\
nem füstöl a tűz, ha lángja mindent felemészt.
}{
\emph{Abolfazl Rashidoddin Meybodi} (XII.~sz.~eleje)
szunnita szúfi tudós. Életéről keveset tudni,
de fennmaradt fő műve, egy monumentális Korán-kommentár.
}
\poem{Mahsati Ganjavi}{مهستی گنجوی}{
چشمم چو به چشم خویش چشم تو بدید & بی چشم تو خواب چشمم از چشم پرید\\
ای چشم همه چشم به چشمت روشن & چون چشم تو چشم من دگر چشم ندید
}{
cheshm-am cho be cheshm-e khish cheshm-ē to bedid\\
bi cheshm-e to khâb-e cheshm-am az cheshm parid\\
ey cheshm hamē cheshm be cheshm-et rowshan\\
chon cheshm-e to cheshm-e man degar cheshm nadid
}{
Láttam szemedet saját szememmel, s emiatt\\
estére szememből álmom messze szaladt.\\
Ó szem, mire minden szemnek fénye vetül,\\
nincs más szempár sehol, szemem min megakad.
}{
\emph{Mahsati} ,,Hold-asszony'' (1089{--}1175?) azerbajdzsáni költőnő.
Szókimondó versei és szabadelvű élete miatt sok üldöztetésben volt része.
}
\poem{`Eynolqozât Hamadâni}{عین‌القضات همدانی}{
ناگه ز درم درآمد آن دلبر مست & جام می لعل نوش کرد و بنشست\\
از دیدن و از گرفتن زلف چو شست & رویم همه چشم گشت و چشمم همه دست
}{
nâgah ze dar-am dar-âmad ân delbar-e mast\\
jâm-ē mey-e la`l nush kard-ō benshast\\
az didan-o az gereftan-ē zolf cho shast\\
ruy-am hame cheshm gasht-o cheshm-am hame dast
}{
Ő hirtelen ajtóm küszöbén átlépett\\
s bort íva magához húzott egy széket.\\
Gyűrűző fürtjét hogy lássam s fogjam,\\
arcom csupa szem, szemem pedig mind kéz lett.
}{
\emph{`Eynolqozât} ,,bírók gyöngye'' \emph{Hamadâni} (1098{--}1131) jogász,
szúfi filozófus, költő és matematikus, \emph{`Omar Khayyâm} tanítványa.
Eretnekségért 33 éves korában kivégezték.
}
\poem{Rashidoddin Vatvât}{رشید‌الدین وطواط}{\label{Vatvat}
بویت شنوم ز باد بی‌هوش شوم & نامت شنوم ز خلق مدهوش شوم\\
اول سخنم توئی چو در حرف آیم & واندیشهٔ من توئی چو خاموش شوم
}{
buy-et shenavam ze bâd bihush shavam\\
nâm-et shenavam ze khalq madhush shavam\\
avval sokhan-am to-i cho dar harf âyam\\
v-andishe-ye man to-i cho khâmush shavam
}{
Szél illatodat ha hozza, mámorba esem;\\
mástól nevedet ha hallom, elvesztem eszem.\\
Szólok -- s első szavam te vagy; nem szólok --\\
s csak téged idéz némán emlékezetem.
}{
\emph{Rashidoddin Vatvât} (1114?{--}1182) szunnita
költő a mai Afganisztán területéről.
A Hvárezmi Birodalomban udvari költőként szolgált.
}
\poem{Ruzbehân Baqli}{روزبهان بقلی}{
دی آینهٔ خویش به صیقل دادم & روشن کردم به پیش خود بنهادم\\
در آینه عیب خویش چندان دیدم & کز عیب کسان هیچ نیامد یادم
}{
di âyene-yē khish be seyqal dâdam\\
rowshan kardam be pish-e khod benhâdam\\
dar 'âyene `eyb-e khish chandân didam\\
k-az `eyb-e kasân hich nayâmad yâd-am
}{
Tegnap lecsiszoltam ezt a tükröt szépen,\\
míg fényes nem lett, s ahogy állt egy széken,\\
rápillantván annyi hibámat láttam,\\
hogy nem jut eszembe most a másé éppen.
}{
\emph{Ruzbehân Baqli} (1128{--}1209) szúfi misztikus filozófus és költő.
Leghíresebb művét, egy látomásokkal teli napló-önéletrajzot,
\emph{A titkok felfedésé\/}t, arabul írta. (1943-ban Khomeyni ugyanezen
a címen publikálta először politikai nézeteit.)
}
\poem{Attâr Neyshâburi}{عطار نیشابوری}{
بازی بودم پریده از عالم راز & تا بو که برم ز شیب صیدی به فراز\\
اینجا چه نیافتم کسی‌را دمساز & زان در که بیآمدم برون رفتم باز
}{
bâz-i budam paride az `âlam-e râz\\
tâ bu ke baram ze shib seyd-i be farâz\\
injâ che nayâftam kas-i-râ damsâz\\
z-ân dar ke biâmadam berun raftam bâz
}{
Titkok völgyéből, mint egy héjamadár,\\
zsákmányommal magasba felszálltam már;\\
ám most a kapun, melyen bejöttem, kimegyek,\\
mert nem leltem barátot, itt senki se vár.
}{
\emph{Attâr} ,,gyógyfüves'' \emph{Neyshâburi} (1145{--}1221) költő,
szúfi gondolkodó és hagiográfus. Gyógyszerészként dolgozott, majd
üzletét otthagyva utazni kezdett, eljutott Mekkába és Indiába is.
Munkássága nagy hatással volt \emph{Rumi\/} költészetére.
(Ezt a verset más forrásokban \emph{Rumi\/}nak tulajdonítják.)\\
Fő művét, \emph{A madarak tanácskozásá\/}t, magyarra is lefordították.
}
\poem{Majdoddin Baqdâdi}{مجدالدین بغدادی}{
شمع است رخ خوب تو پروانه منم & دل خویش غم تو است و بیگانه منم\\
زنجیر سر زلف که بر گردن تست & در گردن من فکن که دیوانه منم
}{
sham`-ast rokh-ē khub-e to parvâne man-am\\
del khish-e qam-ē to-ast-o bigâne man-am\\
zanjir-e sar-ē zolf ke bar gardan-e to-st\\
dar gardan-e man fakan ke divâne man-am
}{
Szép arcod gyertyaláng, e fényhez repked\\
szívem, mint egy moly, idegenként, s szenved;\\
dobd hát a nyakadról hajfürtöd láncát\\
béklyóként rám, ki érted őrülten eped.
}{
\emph{Majdoddin Baqdâdi} (1158{--}1211) orvos,
a \emph{kobrawi} szúfi rend alapítójának,
\emph{Najmoddin Kobrá\/}nak legkedvesebb tanítványa,
és feltehetőleg a híres költő, \emph{Attâr} mestere.
Elsősorban teológiai tárgyú műveket írt, perzsául és arabul is,
leghíresebb közülük a \emph{Resâle dar safar} (Levél az utazásról).
A sah anyjával való kapcsolata miatt kivégezték:
beledobták az \emph{Âmudaryâ} (görög \emph{Oxos}) folyóba.
}
\poem{Kamâloddin Esmâ`il}{کمال‌الدین اسماعیل}{
هر شب ز غمت تو ای نگارین یارم & وز مهر رخ چو ماهت ای دلدارم\\
تا وقت سحر به ماه در می‌نگرم & وز دیده ستارگان فرو می‌بارم
}{
har shab ze qam-et to ey negârin yâr-am\\
v-az mehr-ē rokh cho mâh-et ey deldâr-am\\
tâ vaqt-e sahar be mâh dar-minegaram\\
v-az dide setâregân foru mibâram
}{
Szép kedvesem, arra vágyom, éljél te velem:\\
hold-arcod az ok, hogy éjjel én rendszeresen\\
hajnalhasadásig csak a Holdat nézem,\\
és csillagokat záporozik tőle szemem.
}{
\emph{Kamâloddin Esmâ`il} (1173{--}1237) költő,
elsősorban ódáiról híres.
}
\poem{Rumi}{رومی}{\label{Rumi}
خورشید و ستارگان و بدر ما اوست & بستان و سرای و صحن و صدر ما اوست\\
هم قبله و هم روزه و صبر ما اوست & عید و رمضان و شب قدر ما اوست
}{
khorshid-o setâregân-o badr-ē mâ u-st\\
bostân-o sarây-o sahn-o sadr-ē mâ u-st\\
ham qeble-o ham ruze-o sabr-ē mâ u-st\\
`id-ō ramazân-ō shab-e qadr-ē mâ u-st
}{
Nap, csillagok és Hold, tündöklő -- ez is Ő.\\
Udvar, kert, otthon, s asztalfő -- ez is Ő.\\
Böjt és türelem, kibla, a szent kő -- ez is Ő.\\
Kadr-éj Ramadánkor, ó, ha eljő -- ez is Ő.
}{
\emph{Rumi\/} (1207{--}1273) szúfi teológus és költő. Versei
meghatározó hatással voltak nemcsak a perzsa,
hanem a környező népek irodalmára nézve is.
Műveit számos nyelvre lefordították; népszerűségét mutatja, hogy
gyakran csak úgy hivatkoznak rá, hogy \emph{Mowlânâ} ,,u\-runk''.
Költeményeiben írt törökül, arabul és görögül is.
}
\poem{Sa`di}{سعدی}{
آن یار که عهد دوستداری بشکست & می‌رفت و منش گرفته دامان در دست\\
می‌گفت دگر باره به خوابم بینی & پنداشت که بعد از آن مرا خوابی هست
}{
ân yâr ke `ahd-e dustdâri beshekast\\
miraft-o man-esh gerefte dâmân dar dast\\
migoft degar bâre be khâb-am bini\\
pendâsht ke ba`d az ân ma-râ khâb-i hast
}{
Egy nap, mikor esküvéseit szegve a párom\\
elment, szólt, hogy nem érdemes rá várnom:\\
,,Álmodban fogsz csak engem újból látni!''\\
Csak nem hiszi, hogy szememre jön még álom?
}{
\emph{Sa`di} (1210{--}1291) író, költő.
Prózai művei, mint a \emph{Gyümölcsöskert} és a \emph{Rózsakert}
az egyszerű, de elegáns, kifinomult stílus mintapéldái.
Fiatalkorában sokat utazott, harcolt a kereszteslovagok ellen,
és hét évet fogságban töltött Akkóban.
}
\poemnotoc{Sa`di}{سعدی}{
آنان که پری‌روی و شکرگفتارند & حیفست که روی خوب پنهان دارند\\
فی‌الجمله نقاب نیز بی‌فایده نیست & تا زشت بپوشند و نکو بگزارند
}{
ânân ke pari-ruy-o shekar-goftâr-and\\
heyf-ast ke ruy-e khub penhân dârand\\
feljomle neqâb niz bi-fâyede nist\\
tâ zesht bepushand-o neku bogzârand
}{
Mily kár, hogy a sok tündér utcára ha lép,\\
arcát mind elrejti, s nem látja a nép.\\
Van haszna a fátyolnak azért mégis tán:\\
hogy felvegye az, ki csúnya, s eldobja, ki szép.
}{
}
\poem{`Alâ'oddowle Semnâni}{علاءالدوله سمنانی}{
وقت سحری در‌آمد آن مهرویم & در گوش دلم گفت که ای دلجویم\\
تو هیچ مباش تا همه من باشم & تو هیچ مگوی تا همه من گویم
}{
vaqt-ē sahar-i dar-âmad ân mahru-yam\\
dar gush-e del-am goft ke ey delju-yam\\
tō hich mabâsh tâ hamē man bâsham\\
tō hich maguy tâ hamē man guyam
}{
Hold-arcú kedvesem szobámban jár-kel,\\
szívemhez szól, fülembe súg, így ráz fel:\\
,,Légy semmi, hogy én váljak a mindenséggé;\\
s hogy mindent elmondjak, hallgassál el.''
}{
\emph{`Alâ'oddowle Semnâni} (1261{--}1336) szúfi filozófus
a \emph{kobraviye} rendből.
}
\poemnotoc{`Alâ'oddowle Semnâni}{علاءالدوله سمنانی}{
صد خانه اگر به طاعت آباد کنی & به زان نبود که خاطری شاد کنی\\
گر بنده کنی ز لطف آزادی‌را & بهتر که هزار بنده آزاد کنی
}{
sad khâne agar be tâ`at âbâd koni\\
beh z-ân nabovad ke khâter-i shâd koni\\
gar bande koni ze lotf 'âzad-i-râ\\
behtar ke hazâr bande âzâd koni
}{
Építhetsz szentélyt, száz fajtát-formát:\\
egy kedves szócska többet ér majd odaát.\\
Jobb egy szabad embert leigáznod keggyel,\\
mint felszabadítanod ezer rabszolgát.
}{
}
\poem{Hâfez}{حافظ}{
چون جامه ز تن برکشد آن مشکین خال & ماهی که نظیر خود ندارد به کمال\\
در سینه ز نازکی دلش بتوان دید & مانندهٔ سنگ خاره در آب زلال  
}{
chon jâme ze tan bar-kashad ân moshkin khâl\\
mâh-i ke nazir-e khod nadârad be kamâl\\
dar sine ze nâzoki del-esh betvân did\\
mânande-ye sang-e khâre dar 'âb-e zolâl
}{
Mint páratlan szépségű hold, olyan ő,\\
arcán anyajeggyel, ha levetkőzik e nő.\\
Mellén, bizony oly törékeny, átsejlik a szív,\\
mint kristálytiszta vízen egy gránitkő.
}{
\emph{Hâfez} (1315{--}1390) költészetét a perzsa irodalom csúcsának tekintik;
versgyűjteménye (a Korán mellett) gyakori eleme az újévi dekorációnak.
Szerelmes versei értelmezhetőek szó szerint és szimbolikusan is,
ami nem kevés vitára adott okot.
\emph{Divân\/}jából -- ahogy régen Vergilius műveiből -- ma is gyakran jósolnak
a perzsa nyelvterületeken.
}
\poem{Shâh Ne`matollâh Vali}{شاه نعمت‌الله ولی}{
صبح و سحر و بلبل و گلزار یکیست & معشوقه و عشق و عاشق و یار یکیست\\
هر چند درون خانه را می‌نگرم & خود دایره و نقطه و پرگار یکیست
}{
sobh-ō sahar-ō bolbol-o golzâr yek-i-st\\
ma`shuqe-o `eshq-o `âsheq-ō yâr yek-i-st\\
har chand darun-e khâne-râ minegaram\\
khod dâyere-ō noqte-o pargâr yek-i-st
}{
Éj, hajnal, bülbül és virágtő -- mind egy;\\
vágy és szerelem, barát-barátnő -- mind egy.\\
Bárhányszor nézem én a Házat s mit rejt:\\
Én és a Kör és a Pont s a Körző -- mind egy.
}{
\emph{Shâh Ne`matollâh Vali} (1330{--}1431?) szúfi szent,
a \emph{ne`matollâhi} rend alapítója. Sírja zarándokhely \emph{Mâhân\/}ban.
A versben a Pont (vagy ,,a Kör pontja'') Mohamedre utal; a Kör a világ, a Körző a Teremtő.
}
\poem{Abolvafâ Khârazmi}{ابو‌الوفا خوارزمی}{
بر عقل چو کشف پرده‌ها بود محال & عقل از پس پرده کرد از عشق سؤال\\
تا هست رونده هستی اوست حجاب & ور نیست شود که بهره یابد ز وصال
}{
bar `aql cho kashf-e pardehâ bud mahâl\\
`aql az pas-e parde kard 'az `eshq soâl\\
tâ hast ravande hasti-yē u-st hejâb\\
v-ar nist shavad ke bahre yâbad ze vasâl
}{
Nem képes Ész a leplen átlátni, s ezért\\
függöny mögül ő útmutatást Vágytól kért:\\
,,Falként fedi el léte a vándort, de ha ő\\
nincsen, a haszon kié, ha céljához elért?''
}{
\emph{Abolvafâ Khârazmi} (?{--}1431) szúfi tanító, költő.
}
\poemnotoc{Abolvafâ Khârazmi}{ابو‌الوفا خوارزمی}{
آمد بر من خیال او نیم شبان & گفتم که نثار پای تست این دل و جان\\
گفتا چه دل و چه جان ترا ملک کجاست & آسان باشد سخاوت از مال کسان
}{
âmad bar man khayâl-e u nim-e shabân\\
goftam ke nesâr-e pâ-ye to-st in del-o jân\\
goftâ che del-ō che jân to-râ melk kojâ-st\\
âsân bâshad sakhâvat az mâl-e kasân
}{
Álomkép jött el éjjel, ébenfa-hajú,\\
mondtam, hogy ,,E szívet s lelket vidd, te fiú!''\\
Szólt: ,,Mely szív és lélek? Hol van, mi tiéd?\\
Könnyen vagy a másokéval ily nagyvonalú.''
}{
}
\poem{Jâmi}{جامی}{
دیدار تو ای یار پسندیدهٔ من & حیف است بدین دیدهٔ غمدیدهٔ من\\
در دیدهٔ من نشین و بغشای نقاب & خود بین رخ خویش لیکن از دیدهٔ من
}{
didâr-e to ey yâr-e pasandide-ye man\\
heyf-ast bed-in dide-ye qamdide-ye man\\
dar dide-ye man neshin va-bogshây neqâb\\
khod bin rokh-e khish liken az dide-ye man
}{
Szépséged, kedves, kivetett rám sarcot,\\
csüggedt szemeim már nem akarnak harcot.\\
Ülj hát a szemembe, vesd le kendőd, s te magad\\
bámuld a szemem lencséjén át arcod.
}{
\emph{Jâmi\/}t (1414{--}1492) a középkori perzsa irodalom utolsó nagy költőjeként tartják számon.
Verses és prózai művei mellett számos filozófiai, teológiai és nyelvészeti tárgyú könyvet is írt.
}
\poem{Feyzi}{فیضی}{
آن نیست که ما ارض و سما نشناسیم & تیر قدر و راز قضا نشناسیم\\
این هژده هزار عالم و هر چه در اوست & نشناخته به اگر ترا نشناسیم
}{
ân nist ke mâ arz-o samâ nashnâsim\\
tir-ē qadr-ō râz-e qazâ nashnâsim\\
in hezhde-hazâr `âlam-ō har che dar u-st\\
nashnâkhte beh 'agar to-râ nashnâsim
}{
Számomra a Föld, s az Ég a Földnek peremén,\\
titkot nem rejt, tudom, mi bűn és mi erény.\\
Inkább sokezer világot, és benne mi van,\\
mindent feledek, de téged ismerjelek én.
}{
\emph{Feyzi} (1547{--}1595) indiai költő, Akbar történetírójának bátyja.
Ő fordította le perzsára a híres XII.~századi szanszkrit matematikai művet,
a Līlāvatī-t. Arabul is írt.
}
\poem{Sheykh Bahâi}{شیخ بهای}{
در میکده دوش زاهدی دیدم مست & تسبـیح به گردن و صراحی در دست\\
گفتم ز چه در میکده جا کردی گفت & از میکده هم به سوی حق راهی هست
}{
dar meykade dush zâhed-i didam mast\\
tasbih be gardan-ō sorâhi dar dast\\
goftam ze che dar meykade jâ kardi goft\\
az meykade ham be su-ye haq râh-i hast
}{
Kocsmában a pultra részeg aszkéta borult,\\
és ujjai közt rózsafüzérhez kupa bújt.\\
Megkérdeztem: ,,Te mit csinálsz itt?'', s így szólt:\\
,,Istenhez még kocsmából is vezet út.''
}{
\emph{Sheykh Bahâi} (1547{--}1621) arab származású filozófus, építész, matematikus és csillagász,
az \emph{esfahân\/}i filozófiai iskola alapítója.
}
\poem{Sarmad Kâshâni}{سرمد کاشانی}{
راضی دل دیوانه به تقدیر نشد & فارغ ز خیال فکر و تدبیر نشد\\
ایام شباب رفت و باقیست هوس & ما پیر شدیم و آرزو پیر نشد
}{
râzi del-e divâne be taqdir nashod\\
fâreq ze khayâl-e fekr-o tadbir nashod\\
ayyâm-e shabâb raft-o bâqi-st havas\\
mâ pir shodim-o ârzu pir nashod
}{
Ellenzi a szívem mit a sors néki utal,\\
s tervez: várják még nők és étel-ital.\\
Ifjúkorom elszállt; mi maradt, az csak a vágy.\\
Vén lettem, a szenvedély viszont még fiatal.
}{
\emph{Sarmad Kâshâni} (1590{--}1661) örmény zsidó születésű indiai költő.
Készített egy perzsa Tóra-fordítást is, de később kiábrándult a vallásokból,
és -- egy francia utazó leírása szerint -- \emph{faqir\/}ként, meztelenül,
hosszú hajjal és körmökkel járta Delhi utcáit.
}
\poem{Abdolqâder Bidel}{عبدالقادر بیدل}{
عالم غرض‌آلود جنون من و ماست & اینجا عشق هوس نیالوده کجاست\\
فرهادی و مجنونی اگر می‌شنوی & خود این همه نیست حرف و صوت شعراست
}{
`âlam qaraz-âlud-e jonun-ē man-o mâ-st\\
injâ `eshq-ē havas-nayâlude kojâ-st\\
Farhâd-i-o Majnun-i agar mishenavi\\
khod 'in hame nist harf-o sowt-ē sho`arâ-st
}{
Oly mocskos az Én őrületétől a világ!\\
Hol van szerelem, mit nem szennyez be a vágy?\\
Farhádról vagy Madzsnúnról hallasz tán?\\
Csak költők hangja az, üres líraiság.
}{
\emph{Abdolqâder Bidel} (1642{--}1720) török származású indiai szúfi szent, költő;
\emph{Bidel Dehlavi} ,,delhii Bidel'' néven is ismert.
Tizenhat verseskötetében több, mint 140 ezer vers található.
Egyedi stílusa meghatározó volt az indiai iskola költészetében,
és többek közt nagy hatással volt a legnagyobb urdú költőkre,
\emph{Ghālib\/}ra és \emph{Iqbāl\/}ra is.
}
\poem{Moshtâq Esfahâni}{مشتاق اصفهانی}{
غم بی‌حد و درد بی‌شمار و من فرد & یا رب چه کنم که صبر نتوانم کرد\\
یا درد به اندازهٔ طاقت بفرست & یا حوصله‌ای بده به اندازهٔ درد
}{
qam bi-had-o dard bi-shomâr-ō man fard\\
yâ-rab che konam ke sabr natvânam kard\\
yâ dard be andâze-ye tâqat beferest\\
yâ howsele-i bedeh be andâze-ye dard
}{
Tengernyi a gond, a kín temérdek, s én egy!\\
Mondd, mit tegyek, ó, Uram? A tűrés nem megy!\\
Küldj annyit a kínból, mit erőm még elbír,\\
vagy kínhoz mérten adj türelmet -- mindegy!
}{
\emph{Moshtâq Esfahâni} (1690{--}1758) költő az Afsárida-dinasztia
kezdeti időszakából; a perzsa irodalmi
reneszánsz (\emph{bâzgasht-e adabi}), azon belül is
az \emph{esfahân\/}i stílus egyik korai képviselője.
}
\poem{Fo'âd Kermâni}{فؤاد کرمانی}{
پرسید جوانی شبی از حضرت پیر & خواهم که نمیرم و ندانم تدبیر\\
فرمود که مرگ خلق در زندگی است & خواهی که چو من زنده شوی زنده بمیر
}{
porsid javân-i shab-i az hazrat-e pir\\
khâham ke nemiram va-nadânam tadbir\\
farmud ke marg-e khalq dar zendegi-ast\\
khâhi ke cho man zende shavi zende bemir
}{
Kérdezte a Bölcset egy fiú, hangja kevély:\\
,,Vágyam, hogy örökké éljek. Mondd, van esély?''\\
Így szólt: ,,A teremtménynek a létében a vég;\\
halj meg hát élve, hogy te is, mint én, Élj.''
}{
\emph{Fo'âd Kermâni} (1848{--}1938), valódi nevén
\emph{Fathollâh Qodsi Kermâni}, iráni költő,
elsősorban \emph{`âshurâ}-költeményeiről\footnote{Elégiák
a 680-as karbalai ütközetben elesett \emph{Huseyn} imámhoz.} ismert.
Fő műve a \emph{sham`-e jam`} (A gyülekezet gyertyája) c.~verseskötet.
}
\poem{Malekoshshoarâ Bahâr}{ملک‌الشعراء بهار}{
زاقی می‌گفت اگر بمیرد شهباز & من جای کنم به دست شاهان از ناز\\
بلبل بشنید و گفت کای بندهٔ آز & رو لاف مزن با وزغ و موش بساز
}{
zâqi migoft agar bemirad shahbâz\\
man jây konam be dast-e shâhân az nâz\\
bolbol beshenid o goft k-ey bande-ye âz\\
row lâf mazan bâ vazq-ō mush besâz
}{
Azt mondta a szarka: ,,Hogyha elhull a fehér\\
sólyom, karjára engem ültet a vezér.''\\
Hallotta a bülbül: ,,Ó, mohóság rabja,\\
mit hencegsz?'' -- szólt -- ,,Vár a varangy és az egér!''
}{
\emph{Malekoshshoarâ} ,,költők királya'' \emph{Bahâr} (1886{--}1951), valódi nevén
\emph{Mohammadtaqi Bahâr} iráni költő, politikus,
újságíró és történész. A Tehrán Egyetemen irodalmat tanított, és egy irodalmi folyóiratot
is létesített.
Költeményei hagyományos stílusúak, és gyakran
hazafias hangvételűek.
}
\poemnotoc{Malekoshshoarâ Bahâr}{ملک‌الشعراء بهار}{
برخیز که خود را ز غم آزاده کنیم & تا کی طلب روزی ننهاده کنیم\\
آخر که گل ما به سبو خواهد رفت & کن فکر سبوئی که پر از باده کنیم
}{
bar-khiz ke khod-râ ze qam âzâde konim\\
tâ key talab-ē ruzi-ye nanhâde konim\\
âkhar ke gel-ē mâ be sabu khâhad raft\\
kon fekr-e sabu-'i ke por az bâde konim
}{
Kelj fel, hogy a gondtól szabadítsuk szívünk!\\
Íratlan sorsot még meddig keresünk?\\
Korsó lesz majd porhüvelyünkből úgyis,\\
gondolj a köcsögre, melybe bort töltöttünk!
}{
Ez a vers \emph{Hâfez} egy \emph{qazal\/}jának
parafrázisa, melynek első pár sora:
\begin{Verse}
\emph{beshnow in nokte ke khod-râ ze qam âzâde koni\\
khun khori gar talab-ē ruzi-ye nanhâde koni\\
âkherol'amr gel-ē kuze garân khâhi shod\\
hâliyâ fekr-e sabu kon ke por az bâde koni}
\end{Verse}
}
\poem{Saqir Esfahâni}{صغیر اصفهانی}{
ای ساقی ماه‌وش در این ماه صیام & ده باده که روزه بهر من گشت حرام\\
افطار کنم که دیده ز ابرو و رخت & هم شـکل هلال دید و هم ماه تمام
}{
ey sâqi-ye mâh-vash dar in mâh-e siyâm\\
deh bâde ke ruze bahr-e man gasht harâm\\
eftâr konam ke dide z-abru va-rokh-et\\
ham shakl-e helâl did-o ham mâh-e tamâm
}{
Ó, hold-arcú pohárnokom, jöjj, tölts bort!\\
Bár még Ramadán van, de a böjt nem javasolt,\\
mert orcád és szemöldököd formáján\\
már felragyogott nekem a fél- és telihold.
}{
\emph{Mohammad Hoseyn Saqir Esfahâni} (1894{--}1970) iráni költő.
Különösen misztikus verseiről híres,
a próféta családjának (\emph{ahlol-beyt}) költőjeként tartják számon.
}
\poem{Abolqâsem Hâlat}{ابوالقاسم حالت}{
حیوان درنده‌ای که خونخوار بود & بهتر ز امیری که ستمـکار بود\\
کز آن گاهی به جسم آسیب رسد & وز این همه گاه جان در آزار بود
}{
heyvân-e darande-i ke khunkhâr bovad\\
behtar ze amir-i ke setamkâr bovad\\
k-az 'ân gâhi be jesm 'âsib resad\\
v-az 'in hame gâh jân dar âzâr bovad
}{
Egy vérszomjas vadállat is kellemesebb,\\
mint egy zsarnok, bár legyen ádáz véreb.\\
Mert míg az egyiktől olykor szenved a test,\\
állandó lelki kín a másik, nyílt seb.
}{
\emph{Abolqâsem Hâlat} (1913{--}1992) iráni író, költő;
az 1979-ben alakult Iráni Iszlám Köztársaság első himnuszának szövegírója.
Elsősorban szatíráíról híres.

Ez a vers érdekes párhuzamba állítható a kínai Szertartások könyvének
egy jelenetével, melyben Konfuciusz találkozik egy asz-szonnyal,
aki a zsarnoki uralom elől a tigrisekkel teli erdőbe menekült
(\emph{Lǐ jì: Tán gōng xià 193}).
}
\poem{Javâd Nurbakhsh}{جواد نوربخش}{
بی عشق اگرچه زنده‌ای مرداری & بی عشق اگر عزیز باشی خواری\\
بی عشق اگر شهنشهی مسکینی & با عشق اگرچه عاجزی سرداری
}{
bi `eshq 'agarche zende-i mordâr-i\\
bi `eshq 'agar `aziz bâshi khâr-i\\
bi `eshq 'agar shahanshah-i meskin-i\\
bâ `eshq 'agarche `âjez-i sardâr-i
}{
Vágy nélkül az élő is holt, mint mi kimúlt,\\
vágy nélkül nincsen, ki erős -- mind nyomorult.\\
Vágy nélkül bármi nagy király is koldus:\\
vággyal kap a béna is parancsnok-koszorút.
}{
\emph{Javâd Nurbakhsh} (1926{--}2008) 1953-tól haláláig a \emph{ne'matollâhi}
szúfi rend feje (\emph{pir}), pszichiáter. Orvosi tanulmányait a Sorbonne-on végezte;
az 1979-es forradalom után először Amerikába, majd Angliába emigrált. Számos könyvet írt
a szúfi filozófiáról -- többek közt egy 16-kötetes enciklopédiát a szúfi szimbolizmusról --,
és kiadott egy 30-kötetes pszichiátriai kézikönyvet is.
}
\poem{Mahdiyye Elâhi Qomshei}{مهدیه الهی قمشه‌ای}{
هنگام سحر چو یاد روی تو کنم & با هر سر مویم آرزوی تو کنم\\
هر صبح که سر ز خواب خوش بردارم & با دل همه لحظه گفتگوی تو کنم
}{
hangâm-e sahar cho yâd-e ru-yē to konam\\
bâ har sar-e mu-yam ârzu-yē to konam\\
har sobh ke sar ze khâb-e khosh bar-dâram\\
bâ del hame lahze goftogu-yē to konam
}{
Hajnalban az arcod felidézem, s minden\\
hajszálammal vágyok rád őrülten.\\
Reggel, mikor édes álmom elszáll, folyton\\
rólad csevegek szívemmel kettesben.
}{
\emph{Mahdiyye Elâhi Qomshei} (1937{--}2016) iráni költőnő és irodalmár;
elsősorban \emph{Rumi} költészetével foglalkozott.

Apja (\emph{Mahdi Elâhi Qomshei\/})
készítette a legkedveltebb perzsa Korán-fordítást;
öccse, \emph{Hossein}, híres perzsa irodalmár és filozófus.
}
\poem{Mansur Owji}{منصور اوجی}{
ناگاه شبی ز خواب بیدار شدیم & آیـینهٔ راز و رمز اسرار شدیم\\
با بانگ انا الحقی که بر خاک زدیم & منصور شدیم و باز بر دار شدیم
}{
nâgâh shab-i ze khâb bidâr shodim\\
âyine-ye râz-o ramz-e asrâr shodim\\
bâ bâng-e ana-l-haq-i ke bar khâk zadim\\
mansur shodim-o bâz bar dâr shodim
}{
Álmomból felriadva én lettem a Szó:\\
rejtélyek tükre, kulcs, a Titkot kinyitó.\\
,,Isten vagyok én!'' -- szóltam a Földhöz, amitől\\
\emph{Mansur\/}rá váltam és megint vár a bitó.
}{
\emph{Mansur Owji} (1937{--}) iráni költő. A vers egy másik \emph{Mansur\/}ra utal:
\emph{Mansur-e Hallâj} (858{--}922) szúfi misztikus. Híres mondása \emph{'ana l-\d{h}aqq}
,,én vagyok az Igazság/Isten'' (arab), ami miatt halálra ítélték
(bár valószínűleg itt az Istennel való teljes egyesülésre utalt).
}
\poem{Seyyed Mohammad Beheshti}{سید محمد بهشتی}{
دیریست کنار چشمه چشمم نگران & ناگاه مگر ببیـنمت چون دگران\\
پوشی رخ و گویی که گنهکـاری تو & بخشنده توئی گناه خونین جگران
}{
dir-i-st kenâr-e cheshme cheshm-am negarân\\
nâgâh magar bebinam-et chon degarân\\
pushi rokh-o gui ke gonahkâr-i tō\\
bakhshande to-i gonâh-e khunin jegarân
}{
Régóta figyel már a szemem egy patakot,\\
hogy mint mások, megláthatnám alakod.\\
Elrejted az arcod, s bűnösnek nevezel;\\
vérző szívek feloldozását te adod.
}{
\emph{Seyyed Mohammad Beheshti} (1951{--}) iráni építész.
Fontos szerepe volt
a Fajr nemzetközi filmfesztivál, a Farabi filmművészeti alapítvány
és a Turisztikai és Kulturális Örökség Kutatóintézet megalapításában,
illetve az iráni építészeti enciklopédia készítésében.
Számos egyetemen tart előadásokat iráni kultúrtörténetről.
}
