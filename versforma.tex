A \emph{robâi} egy nagyon kötött versforma. A rímképlete $AABA$, tehát
a harmadik sor kivételével mindegyik rímel (néhány versben még a
harmadik is). Így tehát tekinthető egy négysoros \emph{qazal}-nak,
mivel annak rímképlete $AABACADA\dots$ (a borítón egy 16 soros
\emph{qazal} látszik).

A rímelés szabálya egyszerű: a sorok végén egy vagy több hangnak
(köztük legalább egy magánhangzónak)
pontosan meg kell egyeznie. Ez az egyezés tartalmazhat teljes szavakat
is (\emph{refrén}), de mindig kell lennie egy olyan részének is, amely
minden sorban különböző jelentésű. Vegyük például \emph{Rashidoddin
Vatvât} egy versét (\pageref{Vatvat}.~o.). A rímelő sorok mind
a \emph{shavam} refrénre végződnek, előtte pedig a \emph{bihush},
\emph{madhush} és \emph{khâmush} szavak alkotják a rímet. Ezek közt a
pontos egyezés csupán két hang, az \emph{-ush}.

Ez a fajta rímelés időnként kicsit idegen a magyar fülnek, ezért a
fordításokban, bár a rímképletet megtartottam, a magyar rímelés
szabályaihoz igazodtam.

A klasszikus perzsa költészet időmértékes. A versek analizálásához
el kell tudnunk dönteni a szótagok hosszát:

\begin{enumerate}[i)]
  \item \emph{Rövid} szótag az, amelyikben rövid magánhangzó van
    (\emph{a}/\emph{e}/\emph{o} vagy rövidült \emph{i}), és ezt nem
    követi mássalhangzó, pl.~\emph{ze}.
  \item \emph{Hosszú} szótag az, amelyik olyan, mint egy rövid szótag, de
    vagy hosszú magánhangzó van benne (a kettőshangzók is hosszúnak
    számítanak), vagy a magánhangzó ugyan rövid, de azt egyetlen mássalhangzó
    követi. Az első típusra példa a \emph{ru} szó, a másodikra a
    \emph{shab}.
  \item \emph{Nyújtott} szótag az, amelyikben vagy egy hosszú magánhangzó
    után következik mássalhangzó, pl.~\emph{yâr}, vagy
    mássalhangzótorlódást tartalmaz, pl.~\emph{dast}, vagy mindkettő,
    pl.~\emph{mâst}. A nyújtott szótag metrikai szempontból hosszú +
    rövid értékű (\metra{\m\b}).
\end{enumerate}

Van ezen kívül még néhány kivételes eset, amelyekről érdemes szót ejteni.
Egyrészt verssor végén minden szótag hosszúnak számít.
Másrészt egy hosszú magánhangzó után magában álló $n$ hangra végződő
szótag nem lesz nyújtott, így pl.~az \emph{ân} szó hosszúnak számít, míg
az \emph{âb} nyújtottnak. Ettől csak rendkívül ritkán térnek el.

Ha egy mássalhangzóra végződő szót magánhangzóval kezdődő követ, akkor
általában ez utóbbi ,,átvállalja'' az előző szótag utolsó
mássalhangzóját (\emph{liaison}), és ennek megfelelően változik a
szótagok hossza is. Így pl.~a \emph{beh agar} \metra{\m\b\m} helyett
\metra{\b\b\m}, és a \emph{kard az} \metra{\m\b\m} helyett
\metra{\m\m} lesz. Időnként a szavakat mégis szétválasztják, ezt
ilyenkor a második szó előtt egy aposztróffal jelöltem, pl.~\emph{beh
'agar}.

Szintén előfordul, hogy a ritmus kedvéért megnyújtják a rövid \emph{e}
és \emph{o} magánhangzókat. Ez különösen gyakran jelentkezik az `és'
jelentésű \emph{-o} szónál, és a perzsa nyelv jellegzetes
struktúra-összekötő \emph{-e} szuffixumánál, valamint az egyesszám
második személyű személyes névmásnál (\emph{to}). E\-ze\-ket a
változásokat makronnal jeleztem (\emph{ē}/\emph{ō}).

A \emph{robâi} ritmusa minden sorban (egymástól függetlenül) kétféle
lehet:
\begin{center}
  {\large\metra{\m\m\mbb\s\m\m\mbb\s\m\m\mbb\s\m\cc}}\\
  vagy\\
  {\large\metra{\m\m\mbb\s\m\b\m\b\s\m\m\mbb\s\m\cc}}
\end{center}

A fordítások is ezt követik, de természetesen a magyar nyelv
szótaghosszai alapján.

\begin{center}
  \pgfornament[anchor=south,width=20mm]{82}
\end{center}

Érdemes megjegyezni, hogy a perzsa metrikák túlnyomó része leírható
egy nagyon logikus rendszerben.\footnote{F.~Thiesen: \emph{A Manual of
    Classical Persian Prosody -- with chapters on Urdu, Karakhanidic
    and Ottoman prosody}. Otto Harrassowitz, Wiesbaden, 1982.}

Jelöljük számokkal az alábbi lábakat:
\begin{center}
  1.~\metra{\b\m\m}\quad 2.~\metra{\b\m\m\m}\quad 3.~\metra{\b\b\m\m}\quad
  4.~\metra{\b\m\b\m\b\b\m\m}\quad 5.~\metra{\m\m\b\b\m\b\m\b}
\end{center}

Ekkor az $i.j.k$ hármas által jelölt ritmust úgy kapjuk meg, hogy az
$i$-edik láb $j$-edik szótagjától kezdve $k$ szótagot veszünk,
folyamatosan ismételve a lábat, pl.:
\begin{center}
4{.}5{.}11 \metra{\b\b\m\m\s\b\m\b\m\s\b\b\m}
\end{center}
Itt a 4.~láb (\metra{\b\m\b\m\b\b\m\m}) 5.~szótagjától, tehát a második felétől kezdve
számolunk le 11 szótagot, a láb utolsó (8.)~szótagja után a láb elejétől folytatva.

Amennyiben a sor végére rövid szótag esne, ez automatikusan hosszú
lesz, pl.:
\begin{center}
4{.}7{.}7 \metra{\m\m\b\m\b\m\m}
\end{center}

Ezekben a mértékekben a két rövid szótagot (\metra{\b\b})
helyettesítheti egy hosszú (\metra{\m}), kivéve a sor legelején, ahol
egy hosszú és egy rövid (\metra{\m\b}) kerülhet a helyére, így a fenti
példa képlete pontosabban:
\begin{center}
4{.}5{.}11 \metra{\mb\b\m\m\s\b\m\b\m\s\mbb\m}
\end{center}

Ez alapján a \emph{robâi} sorok két variánsa megfelel a 3{.}3{.}13
ill.~5{.}1{.}13 képleteknek.

\bigskip

További gyakori konfigurációk:
\begin{framed}
\begin{tabular}{llll}
1{.}1{.}11 & {} & {} & {} \\
2{.}1{.}11 & 2{.}1{.}16 & 2{.}4{.}11 & 2{.}4{.}15 \\
3{.}1{.}15 & 3{.}3{.}14 & {} & {} \\
4{.}5{.}11 & 4{.}7{.}7$^\dagger$ & 4{.}7{.}14 & {} \\
5{.}1{.}10 & {} & {} & {}
\end{tabular}

$^\dagger$ Duplázva, a két félsor közt cezúrával.
\end{framed}

\medskip

Ezek közül kiemelendő a \emph{Királyok könyvé\/}ben is használt eposzi verselés (1{.}1{.}11), illetve
a páronként rímelő sorokból álló \emph{masnavi} költeményekben alkalmazott 2{.}4{.}11 lüktetés (pl.~\emph{A madarak tanácskozása}).
