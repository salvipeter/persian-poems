A \emph{robâi} egy nagyon kötött versforma. A rímképlete $AABA$, tehát
a harmadik sor kivételével mindegyik rímel (bár időnként még a
harmadik is). Így tehát tekinthető egy négysoros \emph{qazal}-nak,
mivel annak rímképlete $AABACADA\dots$ (a címlapon egy 16 soros
\emph{qazal} látszik).

A rímelés szabálya egyszerű: a sorok végén egy vagy több hangnak
pontosan meg kell egyeznie. Ez az egyezés tartalmazhat teljes szavakat
is (,,refrén''), de mindig kell lennie egy olyan részének is, amely
minden sorban különböző jelentésű. Vegyük például \emph{Rumi} egy versét
(\pageref{Rumi}.~o.). A rímelő sorok mind a \emph{mâ u-st} refrénre
végződnek, előtte pedig a
\emph{badr} / \emph{sadr} / \emph{sabr} / \emph{qadr} szavak alkotják a
rímet. Ezek közt a pontos egyezés itt az \emph{r}-re korlátozódik, ám
ezt még elegánsabbá teszi a szavak összecsengése.

Ez a fajta rímelés kicsit idegen a magyar fülnek, ezért a
fordításokban, bár a rímképletet megtartottam, a magyar rímelés
szabályaihoz igazodtam.

A klasszikus perzsa költészet időmértékes. A versek analizálásához
el kell tudnunk dönteni a szótagok hosszát:

\begin{enumerate}[i)]
  \item Rövid szótag az, amelyikben rövid magánhangzó van
    (\emph{a}/\emph{e}/\emph{o} vagy rövidült \emph{i}), és ezt nem
    követi mássalhangzó, pl.~\emph{ze}.
  \item Hosszú szótag az, amelyik olyan, mint egy rövid szótag, de
    vagy hosszú magánhangzó van benne (a kettőshangzók is hosszúnak
    számítanak), vagy a magánhangzó ugyan rövid, de azt mássalhangzó
    követi. Az első típusra példa a \emph{ru} szó, a másodikra a
    \emph{shab}.
  \item ,,Túlhosszú'' szótag az, amelyikben vagy egy hosszú magánhangzó
    után következik mássalhangzó, pl.~\emph{yâr}, vagy
    mássalhangzótorlódást tartalmaz, pl.~\emph{dast}, vagy mindkettő,
    pl.~\emph{mâst}. A túlhosszú szótag metrikai szempontból hosszú +
    rövid értékű (\metra{\m\b}).
\end{enumerate}

Van ezen kívül még néhány kivételes eset, amelyekről érdemes szót ejteni.
Egyrészt verssor végén minden szótag hosszúnak számít.
Másrészt hosszú magánhangzó utáni, magában álló $n$ hang
nem lesz túlhosszú, így pl.~az \emph{ân} szó hosz-szúnak számít, míg
az \emph{âb} túlhosszúnak. Ettől csak rendkívül ritkán térnek el.

Ha egy mássalhangzóra végződő szót magánhangzóval kezdődő követ, akkor
általában ez utóbbi ,,átvállalja'' az előző szótag utolsó
mássalhangzóját (\emph{liaison}), és ennek megfelelően változik a
szótagok hossza is. Így pl.~a \emph{beh agar} \metra{\m\b\m} helyett
\metra{\b\b\m}, és a \emph{kard az} \metra{\m\b\m} helyett
\metra{\m\m} lesz. Időnként a szavakat mégis szétválasztják, ezt
ilyenkor a második szó előtt egy aposztróffal jelöltem, pl.~\emph{beh
'agar}.

Szintén előfordul, hogy a ritmus kedvéért megnyújtják a rövid \emph{e}
és \emph{o} magánhangzókat. Ez különösen gyakran jelentkezik az ,,és''
jelentésű \emph{-o} szónál, és a perzsa nyelv jellegzetes
struktúra-összekötő \emph{-e} szuffixumánál, valamint az egyesszám
második személyű személyes névmásnál (\emph{to}). E\-ze\-ket a
változásokat makronnal jeleztem (\emph{ē}/\emph{ō}).

A \emph{robâi} ritmusa minden sorban (egymástól függetlenül) kétféle
lehet:
\begin{center}
  {\large\metra{\m\m\mbb\s\m\m\mbb\s\m\m\mbb\s\m\cc}}\\
  vagy\\
  {\large\metra{\m\m\mbb\s\m\b\m\b\s\m\m\mbb\s\m\cc}}
\end{center}

A fordítások is ezt követik, de természetesen a magyar nyelv
szótaghosszai alapján.

\begin{center}
  \pgfornament[anchor=south,width=20mm]{82}
\end{center}

Érdemes megjegyezni, hogy a perzsa metrikák túlnyomó része leírható
egy nagyon logikus rendszerben.\footnote{F.~Thiesen: \emph{A Manual of
    Classical Persian Prosody -- with chapters on Urdu, Karakhanidic
    and Ottoman prosody}. Otto Harrassowitz, 1982.}

Jelöljük számokkal az alábbi lábakat:
\begin{center}
  1.~\metra{\b\m\m}\quad 2.~\metra{\b\m\m\m}\quad 3.~\metra{\b\b\m\m}\quad
  4.~\metra{\b\m\b\m\b\b\m\m}\quad 5.~\metra{\m\m\b\b\m\b\m\b}
\end{center}

Ekkor az $i.j.k$ hármas által jelölt ritmust úgy kapjuk meg, hogy az
$i$-edik láb $j$-edik szótagjától kezdve $k$ szótagot veszünk,
folyamatosan ismételve a lábat, pl.:
\begin{center}
4{.}5{.}11 \metra{\b\b\m\m\s\b\m\b\m\s\b\b\m}
\end{center}

Amennyiben a sor végére rövid szótag esne, ez automatikusan hosszú
lesz, pl.:
\begin{center}
4{.}7{.}7 \metra{\m\m\b\m\b\m\m}
\end{center}

Ezekben a mértékekben a két rövid szótagot (\metra{\b\b})
helyettesítheti egy hosszú (\metra{\m}), kivéve a sor legelején, ahol
egy hosszú és egy rövid (\metra{\m\b}) kerülhet a helyére.

Ez alapján a \emph{robâi} sorok két variánsa megfelel a 3{.}3{.}13
ill.~5{.}1{.}13 képleteknek.

További gyakori konfigurációk:
\begin{itemize}
  \item 1{.}1{.}11 (eposzi verselés, pl.~a Királyok könyve)
  \item 2{.}1{.}11
  \item 2{.}1{.}16 (a címlapon látható \emph{qazal} lüktetése)
  \item 2{.}4{.}11 (\emph{masnavi}, pl. A madarak tanácskozása)
  \item 2{.}4{.}15
  \item 3{.}1{.}15
  \item 3{.}3{.}14
  \item 4{.}1{.}15
  \item 4{.}5{.}11
  \item 4{.}7{.}7 (duplázva, a két félsor közt cezúrával)
  \item 4{.}7{.}14
  \item 5{.}1{.}10
\end{itemize}
