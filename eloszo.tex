A perzsa \emph{robâi} négysoros versek az európai ember számára \emph{Omar
Khayyâm} költeményeit jelentik, pedig Perzsiában ez a körülbelül ezer
éves műfaj állandó népszerűségnek örvendett, és minden ismertebb tudós
és költő alkotott benne.

Ebben a kötetben kiválogattam néhány olyan verset, a\-me\-lyek valami
miatt szerintem különlegesek. Ez lehet egy ügyes szójáték, egy érdekes
gondolat, egy meglepő kép -- valami, ami miatt kilép a vers az erősen
kötött forma és a hagyományos témák keretein kívülre.

A \emph{robâi} versek témája három nagy csoportra osztható:

\begin{enumerate}
\item A földi lét gyors elmúlása, a ma élvezete. Gyakori
  szimbólum a föld és az agyag, amibe porhüvelyünk visszatér, illetve
  a bor és a bort töltő \emph{sâqi}.
\item Szerelmes vers a Kedveshez. Mivel a perzsában a személyes névmás
  nemtől független, az esetek nagy részében csak találgathatunk, hogy
  nőkhöz vagy férfiakhoz íródtak, de feltételezhető, hogy sokat írtak
  fiatal fiúkhoz. Itt gyakorta megjelennek híres szerelmespárok, mint
  az őrült \emph{Majnun} és \emph{Layla}, vagy \emph{Farhâd} és
  \emph{Shirin}. Kedvelt szimbólum a bülbül-madár, aki egész éjjel
  énekel hajthatatlan szerelméhez, a rózsához, vagy az éjjeli lepke,
  akit vonz a gyertya lángja, amiben aztán elég.
\item Szúfi vallásos/filozofikus vers. Ez az előző kategóriától nem mindig válik
  el élesen, mert a szúfi gondolkodásmód szerint Isten a Kedves, aki
  iránt való szerelem, és az abban való egyesülés, az Éntől való
  megszabadulás, a földi lét egyetlen értelme. Itt minden szó új
  jelentést kap -- a bor pl.~általában az igazságot, az Istenhez való
  közelséget jelenti, a kocsma a világ stb. A világmindenség titkait
  kendő fedi, ami mögé csak a kiválasztottak nyernek bepillantást.
\end{enumerate}

A fordítások pontosan tartják az eredeti versformát, néhol a pontosság
kárára. Igyekeztem azonban minél kevésbé eltérni az eredetitől, és
megtartani annak legalább szellemét, ha a pontos szavakat nem is
mindig sikerült.

A perzsa nyelv tanulói számára minden versnél szerepel az eredeti
szöveg és annak egy latin betűs átírása is. Mivel a nyelv az utóbbi
ezer évben keveset változott, azt remélem, hogy a fordítás és egy
szótár segítségével az értelmezés nem jelenthet gondot, és így talán
az eredeti vers élvezetét is sikerül hozzáférhetőbbé tennem.

\begin{flushright}
  Salvi Péter, 2020
\end{flushright}
