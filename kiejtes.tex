A perzsa nyelvnek különböző változatait beszélik ma Iránban,
Afganisztánban és Tádzsikisztánban, és ezeknek kiejtése is
különböző. Noha a dári és a tádzsik jobban megtartotta a klasszikus perzsa
magánhangzórendszert, itt mégis inkább a nemzetközileg ismertebb iráni
perzsa kiejtést ismertetem.

A latin betűs átírás a kettőshangzók írásmódjának kivételével
Thackstont\footnote{W.~M.~Thackston: \emph{An Introduction to Persian},
4th Ed. Ibex Publishers, Bethesda, Maryland, 2009.} követi.

A magyartól eltérő kiejtésű betűket és betűkombinációkat a következő
oldalon található táblázat tartalmazza. Ezekhez további megjegyzések:

\begin{itemize}
  \item A \emph{ch}, \emph{k}, \emph{p}, \emph{t} hangok hehezetesek,
    kiejtésükkor több levegőt kell kifújni, mint a magyarban.
  \item A \emph{g} és \emph{k} hangok szótag végén és az
    \emph{e}/\emph{i}/\emph{a} magánhangzók előtt palatalizálódnak,
    tehát ilyenkor kiejtésüket egy rövid [j] hang követi.
  \item Az \emph{i} más magánhangzók és \emph{y} előtt [i]-re rövidül.
  \item A \emph{q} hang magánhangzók közt hörgő párizsi [r] hangba
    válthat át.
  \item A két kettőshangzó az \emph{ey} és az \emph{ow}, melyek ejtése
    rendre [éj] és [ou].
\end{itemize}

A hangsúly legtöbb esetben a szó végére kerül; a hangsúlytalan
szuffixumokat kötőjellel jeleztem.

\begin{table}[h]
  \begin{center}
    \smallskip\smallskip
    \begin{tabular}{cl}
      \emph{a} & rövid [á] és [e] közti hang\\
      \emph{â} & [aa]\\
      \emph{ch} & [cs]\\
      \emph{e} & [e] és rövid [é] közti hang\\
      \emph{i} & [í]\\
      \emph{j} & [dzs]\\
      \emph{kh} & hörgő [h] hang\\
      \emph{q} & hátul képzett [g] hang\\
      \emph{s} & [sz]\\
      \emph{sh} & [s]\\
      \emph{u} & [ú]\\
      \emph{y} & [j]\\
      \emph{zh} & [zs]\\
      \emph{`} & torokzárás, elválasztja a szótagokat\\
    \end{tabular}
  \end{center}
  \caption{A latin betűs perzsa átírás kiejtése.}
\end{table}
